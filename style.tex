A complete definition of TCA naming conventions and code-guidelines is available at \url{https://192.168.1.221/references/tca-code-guidelines}. It is a subset of the C++-Google Code-Guidelines and differs slightly.

Here the most important should be listed:

\paragraph{Scoping}

\begin{enumerate}
	\item Namespaces:
	\begin{itemize}
		\item With few exceptions, place code in a namespace. Namespaces should have unique names based on the project name, and possibly its path.
		\item Namespaces subdivide the global scope into distinct, named scopes, and so are useful for preventing name collisions in the global scope.
	\end{itemize}

	\item Local Variables:
	\begin{itemize}
		\item Place a function's variables in the narrowest scope possible, and initialize variables in the declaration.
		
		\begin{minted}{cpp}
		int i;
		i = f();      // Bad -- initialization separate from declaration.
		\end{minted}
		
		\begin{minted}{cpp}
		int j = g();  // Good -- declaration has initialization.
		\end{minted}
		
		\begin{minted}{cpp}
		// bad
		std::vector<int> v;
		v.push_back(1);  // Prefer initializing using brace initialization.
		v.push_back(2);
		\end{minted}
		
		\begin{minted}{cpp}
		// good
		std::vector<int> v = {1, 2};  // Good -- v starts initialized.
		\end{minted}
	\end{itemize}
\end{enumerate}

\paragraph{Classes}

\begin{enumerate}
	\item Implicit Conversions:
	\begin{itemize}
		\item Do not define implicit conversions. Use the explicit keyword for conversion operators and single-argument constructors.
	\end{itemize}
	
\end{enumerate}

\todo{naming conventions}
\paragraph{Naming Conventions}

\inputminted{cpp}{src/style-and-guidelines/naming/naming.cpp}
