\subsection{Initialization}

\begin{itemize}
	\item initialize everything with reasonable values
	\item some std classes do not need to be initialized specifically to have default values 
	
	(i.e. \mintinline{cpp}{std::string s; // initializes empty string automatically})
\end{itemize}

\subsection{Constructors \& Destructors}

\paragraph{Constructors}
\begin{itemize}

	\item avoid initialization that can fail if you can't signal an error:
	
	There is no easy way for constructors to signal errors, short of crashing the program (not always appropriate) or using exceptions.
	\item use the constructor of classes to initialize values if possible
	\item use the constructor initializer lists
	\begin{minted}{cpp}
	Auto::Auto(double power, double speed, std::string name)
	 : power_(power)
	 , speed_(speed)
	 , name_(name)
	{}
	\end{minted}
	
	\item in initializer lists the order is important
	\item in C++ 11 initialization of standard types can be done in the header file:
	\begin{minted}{cpp}
	class Auto {
	private:
		double power_ = 0.0;
		double speed_ = 0.0;
		std::string name_ = "Audi R8";
	}
	\end{minted}
	
	\item if no constructor is defined, the compiler provides a standard constructor
	\item for container types (\url{http://www.cplusplus.com/reference/stl/}), a standard constructor has to be defined. The following would not work:
	
	\begin{minipage}[t]{0.4\textwidth}
		\inputminted{cpp}{src/constructors/ex-1/main.cpp}
	\end{minipage}\hfill
	\begin{minipage}[t]{0.4\textwidth}
		\inputminted{cpp}{src/constructors/ex-1/ConstructorClass.h}
	\end{minipage}

	\item as soon as dynamic types (pointers) are part of a class, a self-written constructor, copy-constructor \& destructor is advised to handle the pointer data correctly

	\item special types of constructors:
	\begin{itemize}
		\item copy constructor
		\item move constructor
	\end{itemize}

	\item see rule of three below (section \ref{subsec:rule-of-three})
\end{itemize}

\paragraph{Destructors}

\begin{itemize}
	\item 
	\begin{tcolorbox}[notitle,boxrule=0.5pt,colback=red!20,colframe=black!90]
		\textbf{NOTE:} A destructor should be defined as soon as dynamic types are part of the class. 
	\end{tcolorbox}
\end{itemize}