\documentclass[
10pt,
a4paper,
parskip=half,	% space between parapraphs
DIV=0,			% bigger margins
BCOR=10mm
]{scrartcl}
\usepackage[utf8]{inputenc}
\usepackage[T1]{fontenc}
\usepackage[english]{babel}
\usepackage{a4wide}
\usepackage{amsmath}
\usepackage{amsfonts}
\usepackage{amssymb}
\usepackage{graphicx}
\usepackage{lmodern}
\usepackage{minted}
\usepackage{tabularx}
\usepackage[final]{pdfpages}
\usepackage{framed}
\usepackage{url}
\usepackage[colorinlistoftodos,prependcaption,textsize=tiny]{todonotes}

% tables ------
\usepackage{tabularx}
\usepackage{booktabs}
\usepackage{hhline} % alternative to cline in tables
\usepackage{ltablex} % for tables longer than one page, enables page-break
% -------------

% add space after \paragraph{}
\makeatletter
\renewcommand\paragraph{%
	\@startsection{paragraph}{4}{0mm}%
	{-\baselineskip}%
	{.5\baselineskip}%
	{\normalfont\normalsize\bfseries}}
\makeatother
%-------------------------------------


\author{David Gmeindl}

%------------- begin scrheadings style ---------------------
\makeatletter

\usepackage{scrlayer-scrpage} %%  advanced page style using KOMA
\pagestyle{scrheadings}
\automark[section]{section} % sets headmark
\ihead*{\headmark}
\chead*{}
\ohead*{TeleConsult-Austria}
\ofoot*{\@author}
\KOMAoptions{
	headsepline = true,
	footsepline = true,
	plainfootsepline = true
}

\makeatother
%------------- end scrheadings style -----------------------

% pdf editing -----
\usepackage[pdftex,plainpages=false]{hyperref}
\hypersetup{%
	hidelinks=true,
	colorlinks=false,
	linkcolor=green,
	pdfstartview=FitV} % PDF-Viewer benutzt beim Start bestimmte Seitenbreite
% -----------------

\usepackage{tcolorbox}	% i.e.: \begin{tcolorbox}[breakable,notitle,boxrule=0.5pt,colback=green!20,colframe=black!90] ... \end{tcolorbox}

\pagestyle{empty}
\title{C++ Guidelines}

\newcommand{\ttt}[1]{\texttt{#1}}
\newenvironment{notebox}[1]{
	\begin{tcolorbox}[notitle,boxrule=0.5pt,colframe=black!90]
		#1}
	{\end{tcolorbox}}

% definition of language for inclusion of external files

\setminted[bash]{tabsize=4,linenos,autogobble,frame=lines,breaklines,breakanywhere=true,fontsize=\footnotesize}
\setminted[cpp]{tabsize=4,linenos,autogobble,frame=lines,breaklines,breakanywhere=true,fontsize=\footnotesize}

\begin{document}
	
	
	\pagestyle{empty}
	\maketitle
	\tableofcontents
	%\listoftodos[Notes]
	\thispagestyle{empty}
	
	\newpage
	\pagebreak
	
	\pagestyle{scrheadings}
	
\section{Style \& Guidelines}
\subsection{Naming}
A complete definition of TCA naming conventions is available at \url{https://192.168.1.221/references/tca-code-guidelines}. It is a subset of the C++-Google Code-Guidelines and differs slightly.

Here the most important should be listed:

\todo{naming conventions}
\subsection{DO's}
\begin{itemize}
	\item use std libraries where-ever possible (known and TESTED behavior, programmed to be efficient)
	\item 
\end{itemize}

\subsection{DONT's}

\begin{itemize}
	\item 
\end{itemize}
	
\section{Constructors \& intitializations}
	
\subsection{Initialization}

\begin{itemize}
	\item initialize everything with reasonable values
	\item some std classes do not need to be initialized specifically to have default values 
	
	(i.e. \mintinline{cpp}{std::string s; // initializes empty string automatically})
\end{itemize}

\subsection{Constructors \& Destructors}

\paragraph{Constructors}
\begin{itemize}

	\item avoid initialization that can fail if you can't signal an error:
	
	There is no easy way for constructors to signal errors, short of crashing the program (not always appropriate) or using exceptions.
	\item use the constructor of classes to initialize values if possible
	\item use the constructor initializer lists
	\begin{minted}{cpp}
	Auto::Auto(double power, double speed, std::string name)
	 : power_(power)
	 , speed_(speed)
	 , name_(name)
	{}
	\end{minted}
	
	\item in initializer lists the order is important
	\item in C++ 11 initialization of standard types can be done in the header file:
	\begin{minted}{cpp}
	class Auto {
	private:
		double power_ = 0.0;
		double speed_ = 0.0;
		std::string name_ = "Audi R8";
	}
	\end{minted}
	
	\item if no constructor is defined, the compiler provides a standard constructor
	\item for container types (\url{http://www.cplusplus.com/reference/stl/}), a standard constructor has to be defined. The following would not work:
	
	\begin{minipage}[t]{0.4\textwidth}
		\inputminted{cpp}{src/constructors/ex-1/ConstructorClassFail.h}
	\end{minipage}\hfill
	\begin{minipage}[t]{0.4\textwidth}
		\inputminted{cpp}{src/constructors/ex-1/main-fail.cpp}		
	\end{minipage}
	
	whereas this works:
	
	\begin{minipage}[t]{0.4\textwidth}
		\inputminted{cpp}{src/constructors/ex-1/ConstructorClass.h}
	\end{minipage}\hfill
	\begin{minipage}[t]{0.4\textwidth}
		\inputminted{cpp}{src/constructors/ex-1/main.cpp}		
	\end{minipage}

	\item as soon as dynamic types (pointers) are part of a class, a self-written constructor, copy-constructor \& destructor is advised to handle the pointer data correctly

	\item special types of constructors:
	\begin{itemize}
		\item copy constructor
		\item move constructor
	\end{itemize}

	\item see rule of three below (section \ref{subsec:rule-of-three})
\end{itemize}

\paragraph{Destructors}

\begin{itemize}
	\item 
	\begin{tcolorbox}[notitle,boxrule=0.5pt,colback=red!20,colframe=black!90]
		\textbf{NOTE:} A destructor should be defined as soon as dynamic types are part of the class. 
	\end{tcolorbox}
\end{itemize}

\section{Classes vs Structures}

\begin{itemize}
	\item use a struct only for passive objects that carry data
	\item everything else is a class
	\item \ttt{structs} should be used for passive objects that carry data, and may have associated constants, but lack any functionality other than access/setting the data members. The accessing/setting of fields is done by directly accessing the fields rather than through method invocations. Methods should not provide behavior but should only be used to set up the data members, e.g., constructor, destructor, initialize(), reset(), validate().
	\item If more functionality is required, a \ttt{class} is more appropriate.
	\item Note: naming conventions of members are different (\_ after private member in class)
	\item Classes should always encapsulate as much data as needed to ease the handling with the class from the outside: clean interfaces
\end{itemize}

\newpage
\paragraph{Examples}

\begin{listing}[htbp]
	\begin{minted}{cpp}
	struct Saturation
	{
		bool active      = false;
		double threshold = 1.0e99;
		
		inline bool operator==(const Saturation &rhs) const
		{
			return ((active == rhs.active) && (threshold == rhs.threshold));
		}
		
		inline bool operator!=(const Saturation &rhs) const { return !operator==(rhs); }
	};
	\end{minted}
	\caption{Data container struct}
\end{listing}

\begin{listing}[htbp]
	\inputminted{cpp}{src/classes-structs/ReceiverPath.h}
	\caption{Class}
\end{listing}
	
\pagebreak
\section{Pointers}
\subsection{Definition}
Video: \url{https://www.youtube.com/watch?v=iChalAKXffs}
\subsection{Pointer Operators}
\todo{put this in proper code snippet and comment it}
\begin{itemize}
	\item Pointer Definition: \mintinline{cpp}{double *p = new double(5.3);}
	\item De-Reference Operator: \mintinline{cpp}{double a = *p;}
\end{itemize}
\subsection{Memory Leak \& Segmentation Fault}

\paragraph{Example 1 - Memory Leak}
	
	\begin{listing}[!htbp]
		\begin{minipage}[t]{0.45\textwidth}
			\inputminted{cpp}{src/pointers/ex-1/MyClass.h}
		\end{minipage}\hfill
		\begin{minipage}[t]{0.45\textwidth}
			\inputminted{cpp}{src/pointers/ex-1/MyClass.cpp}
		\end{minipage}
		\caption{Example 1: MyClass.h \& MyClass.cpp}
	\end{listing}

	A pointer in \ttt{MyClass} is created in the constructor (\ttt{MyClass::MyClass()}), but never deleted; as soon as the destructor of \ttt{MyClass} is called (see scope in main), the pointer has to be deleted, but this never happens. On the contrary, if the std::string pointer is deleted in the destructor, p outside the limited scope (line 13) cannot be accessed anymore (see next example).
	
	\inputminted{cpp}{src/pointers/ex-1/main.cpp}
	
	\begin{minted}{bash}
		valgrind --tool=memcheck --leak-check=full ./main
		
		==1485== HEAP SUMMARY:
		==1485==     in use at exit: 32 bytes in 1 blocks
		==1485==   total heap usage: 3 allocs, 2 frees, 73,760 bytes allocated
		==1485== 
		==1485== 32 bytes in 1 blocks are definitely lost in loss record 1 of 1
		==1485==    at 0x4C3017F: operator new(unsigned long) (vg_replace_malloc.c:334)
		==1485==    by 0x108E9F: MyClass::MyClass() (in /home/david/projects/cpp-intro/src/pointers/ex-1/main)
		==1485==    by 0x108D4D: main (in /home/david/projects/cpp-intro/src/pointers/ex-1/main)
		==1485== 
		==1485== LEAK SUMMARY:
		==1485==    definitely lost: 32 bytes in 1 blocks
		==1485==    indirectly lost: 0 bytes in 0 blocks
		==1485==      possibly lost: 0 bytes in 0 blocks
		==1485==    still reachable: 0 bytes in 0 blocks
		==1485==         suppressed: 0 bytes in 0 blocks
	\end{minted}
	
	\paragraph{Example 2 - Segmentation Fault}
	
	To produce no memory leaks, the allocated memory for \ttt{std::string} is cleaned in the destructor of \ttt{MyClass}. However, a segfault is created, as soon as the pointer points to memory, where no data is assigned anymore (see example below);
	
	\begin{listing}[!htbp]
		\begin{minipage}[t]{0.45\textwidth}
			\inputminted{cpp}{src/pointers/ex-2/MyClass.h}
		\end{minipage}\hfill
		\begin{minipage}[t]{0.45\textwidth}
			\inputminted{cpp}{src/pointers/ex-2/MyClass.cpp}
		\end{minipage}
		\caption{Example 2: MyClass.h \& MyClass.cpp}
	\end{listing}

	\inputminted{cpp}{src/pointers/ex-2/main.cpp}
	
	\begin{minted}{bash}
	david@david-pc:~/projects/cpp-intro/src/pointers/ex-2 $ ./main 
	abc
	Segmentation fault (core dumped)
	\end{minted}
	
	\paragraph{Further Information}
	
	\url{http://www.learncpp.com/cpp-tutorial/69-dynamic-memory-allocation-with-new-and-delete/}
	
\section{Smart Pointers}
    see \url{https://www.acodersjourney.com/top-10-dumb-mistakes-avoid-c-11-smart-pointers/}
    
\section{Const as keyword}
\begin{itemize}
	\item 
\end{itemize}
\section{Programming idioms} 
\label{sec:programming-idioms}
	\subsection{Resource acquisition is initialization (RAII)}
	
	\url{https://en.wikipedia.org/wiki/Resource_acquisition_is_initialization}
	
	\subsection{Rule of three (five, zero)}
	\label{subsec:rule-of-three}
	
	\url{https://en.cppreference.com/w/cpp/language/rule_of_three}
	

\section{Std libraries}
	
	\url{http://www.cplusplus.com/reference/std/}
	
	\paragraph{algorithm}
	
	i.e. \ttt{find\_if}
	\inputminted{cpp}{src/algorithm/ex-1/main.cpp}
	
	i.e. \ttt{unique}
	\inputminted{cpp}{src/algorithm/ex-2/main.cpp}
	
	
\section{General Advice}

\subsection{Do's}
\begin{itemize}
	\item use std libraries where-ever possible (known and TESTED behavior, programmed to be efficient)
	\item 
\end{itemize}

\subsection{Dont's}
\begin{itemize}
	\item 
\end{itemize}

\section{Online \& Offline References}

	\begin{itemize}
		\item C++ Core Guidelines 
		
		\url{https://github.com/isocpp/CppCoreGuidelines/blob/master/CppCoreGuidelines.md}
		
		\item Scott Meyers More Effective C++
		
		\url{https://github.com/vpreethamkashyap/Library/blob/master/Scott%20Meyers-Effective%20Modern%20C%2B%2B_%2042%20Specific%20Ways%20to%20Improve%20Your%20Use%20of%20C%2B%2B11%20and%20C%2B%2B14-O'Reilly%20Media%20(2014).pdf}
	\end{itemize}
	
\end{document}
